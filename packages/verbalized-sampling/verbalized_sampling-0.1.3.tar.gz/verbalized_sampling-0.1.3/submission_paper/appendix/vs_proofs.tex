
% \appendix

% \section{Proof of Modal Type Recovery}
% \label{app:vs-proof}

% \derek{This section is safe to remove, but perhaps best to leave until Anthony gets through as a reference}

% We provide the complete derivation showing that the most likely type under i.i.d. sampling minimizes KL divergence from the base distribution.

% \subsection{Setup and Multinomial Probability}

% Consider outcomes $i = 1, \ldots, m$ with base distribution $p = (p_1, \ldots, p_m) \in \Delta_m$. For a size-$k$ unordered set drawn i.i.d. from $p$ with counts $n = (n_1, \ldots, n_m)$ where $\sum_i n_i = k$, the empirical distribution (type) is $q_i = n_i/k$. The probability of observing type $q$ is:
% \begin{equation}
% \Pr\{\text{type} = q\} = \frac{k!}{\prod_{i=1}^m (kq_i)!} \prod_{i=1}^m p_i^{kq_i}
% \label{eq:app-multinomial}
% \end{equation}

% \subsection{Derivation via Stirling's Approximation}

% Taking logarithms of \eqref{eq:app-multinomial}:
% \begin{equation}
% \log \Pr\{\text{type} = q\} = \log k! - \sum_{i=1}^m \log(kq_i)! + k\sum_{i=1}^m q_i \log p_i
% \end{equation}

% Applying Stirling's approximation $\log n! = n\log n - n + \tfrac{1}{2}\log(2\pi n) + r_n$ with $|r_n| \leq \frac{1}{12n}$:
% \begin{align}
% \log k! &= k\log k - k + \tfrac{1}{2}\log(2\pi k) + r_k\\
% \log(kq_i)! &= kq_i \log(kq_i) - kq_i + \tfrac{1}{2}\log(2\pi kq_i) + r_{kq_i} \quad (q_i > 0)
% \end{align}

% For the main $k$-order terms, noting that $\sum_i kq_i \log(kq_i) = k\log k + k\sum_i q_i\log q_i$:
% \begin{align}
% (k\log k - k) - \sum_i(kq_i \log(kq_i) - kq_i) &= -k\sum_i q_i \log q_i = k H(q)
% \end{align}

% where $H(q) = -\sum_i q_i\log q_i$ is the Shannon entropy. Combined with the cross-entropy term:
% \begin{equation}
% k H(q) + k\sum_i q_i\log p_i = -k \cdot \text{KL}(q\|p)
% \end{equation}

% For the remainder terms, the half-logarithm contribution is:
% \begin{align}
% R_{\text{half-log}}(q) &= \tfrac{1}{2}\left[\log(2\pi k) - \sum_{i:q_i>0}\log(2\pi kq_i)\right]\\
% &= -\tfrac{1}{2}\sum_{i:q_i>0}\log q_i + \tfrac{1}{2}(1-m_+)\log(2\pi k)
% \end{align}

% where $m_+ = |\{i : q_i > 0\}|$. Since $q_i \geq 1/k$ for nonzero coordinates, $|-\tfrac{1}{2}\sum\log q_i| \leq \tfrac{1}{2}m\log k$. Including Stirling remainders, the total remainder is $R_k(q) = O(\log k)$ uniformly over all types.

% Therefore:
% \begin{equation}
% \boxed{\log \Pr\{\text{type} = q\} = -k \cdot \text{KL}(q\|p) + O(\log k)}
% \end{equation}

% Since the $O(\log k)$ term is lower-order, maximizing $\Pr\{\text{type} = q\}$ over $q \in \Delta_m$ is equivalent to minimizing $\text{KL}(q\|p)$. The unique minimizer is $q^* = p$, proving that the most likely type matches the base distribution.

% \subsection{Integer Constraints and Rounding}
% \label{app:rounding}

% In practice, counts must be integers. We characterize the exact modal counts when rounding is necessary.

% \begin{theorem}[Modal Multinomial Counts]
% Let $p \in \Delta_m$ and $k \in \mathbb{N}$. Define $a_i = (k+1)p_i$, $b_i = \lfloor a_i \rfloor$, $f_i = \{a_i\}$ (fractional part), and $B = \sum_i b_i$. The modal count vectors $n^*$ are:

% \begin{itemize}
% \item \textbf{Case 1 ($B = k$):} Unique mode with $n^*_i = b_i$ for all $i$.
% \item \textbf{Case 2 ($B < k$):} Modes have $n^*_i \in \{b_i, b_i+1\}$ with $\sum_i n^*_i = k$, where the $k-B$ increments go to indices with largest $f_i$.
% \item \textbf{Case 3 ($B > k$):} Modes have $n^*_i \in \{b_i-1, b_i\}$ with $\sum_i n^*_i = k$, where the $B-k$ decrements go to indices with smallest $f_i$.
% \end{itemize}

% Equivalently, every coordinate satisfies:
% \begin{equation}
% \lfloor (k+1)p_i \rfloor - 1 \leq n^*_i \leq \lfloor (k+1)p_i \rfloor
% \end{equation}
% \end{theorem}

% \begin{proof}[Proof Sketch]
% For feasible $n$ and any $i \neq j$ with $n_j \geq 1$, moving one count from $j$ to $i$ yields:
% \begin{equation}
% \frac{P(n + e_i - e_j)}{P(n)} = \frac{n_j}{n_i+1} \cdot \frac{p_i}{p_j}
% \end{equation}

% At a maximum $n^*$, this ratio must be $\leq 1$ for all valid $(i,j)$, implying:
% \begin{equation}
% \frac{n^*_j}{p_j} \leq \frac{n^*_i + 1}{p_i} \quad \forall i,j \text{ with } n^*_j > 0
% \end{equation}

% This necessitates existence of $t \in [k, k+1]$ with $tp_i - 1 \leq n^*_i \leq tp_i$ for all $i$. The tightest such window uses $t = k+1$, yielding the stated bounds. The three cases follow from the sum constraint $\sum_i n^*_i = k$.
% \end{proof}


\section{Typicality Bias Causes Mode Collapse}\label{appendix:additional_proof}

\subsection{Typicality Bias in \textsc{HelpSteer}: Experimental Validation Detail}
\label{app:evidence-controls}

% DC: I'm inserting the raw result tables here to start - progressive enhancement into proper figures and tables coming!







% Extended version of main paper result: May be unpacked

% \paragraph{(i) Controlled regression (HelpSteer).} 

As outlined in \cref{sec:mc-typicality}, we test the ``typicality bias'' hypothesis on the training split of \textsc{HelpSteer}~\citep{wang2023helpsteer}. We use per-response ratings for \emph{correctness} and \emph{overall helpfulness} to form $6{,}874$ within-prompt pairs matched on correctness (i.e., $\Delta\text{correctness}=0$), and compute per-token log-likelihoods under two base models, $\pi_{\text{ref}}$: \emph{Llama~3.1~405B Base} and \emph{GLM~4.5 Base}. We then fit the Bradley–Terry logistic model implied by \eqref{eq:bt-assumption}, with the binary outcome ``which response receives higher helpfulness'' and predictor $\Delta\bar{\ell}=\bar{\ell}_i-\bar{\ell}_j$ (difference in average log-likelihood under $\pi_{\text{ref}}$). The coefficient on $\Delta\bar{\ell}$ is the estimate of $\alpha$. Results are provided in Table~\ref{tab:mc-alpha}.



On the correctness-matched pairs, we obtain $\hat{\alpha}=0.57\pm0.07$ for Llama~3.1~Base and $\hat{\alpha}=0.65\pm0.07$ for GLM~4.5~Base (cluster-robust SEs; both $p<10^{-14}$). Interpreted as odds ratios per one standard deviation in $\Delta\bar{\ell}$, this corresponds to $1.42$-$1.47\times$ higher odds of the more typical response being judged more helpful, a 17-19 percentage point increase in win probability. Using all $28{,}283$ within-prompt pairs and adding $\Delta\text{correctness}$ as a covariate yields similar but slightly smaller effects ($\hat{\alpha}\approx0.46$–$0.49$), confirming that the typicality bias predicts helpfulness \emph{above and beyond} correctness. These results provide empirical evidence for a positive $\alpha$ term in \eqref{eq:bt-assumption}, i.e., human annotators reward base-model typicality independent of semantic correctness.

\begin{table}[htbp!]
\centering
\small
\caption{Bradley–Terry regressions estimating the typicality weight $\alpha$. OR = odds ratio per 1 SD of $\Delta\log p$ (base model log-probability). $\Delta P$ = predicted change in win probability from -1 SD to +1 SD.}
\begin{tabular}{lcccccc}
\toprule
Base Model & Slice & $\hat\alpha$ & SE & OR (per 1 SD) & $\Delta P$ (-1→+1 SD) & $N$ pairs \\
\midrule
Llama 3.1 405B & Tie ($\Delta$corr=0) & 0.569 & 0.073 & 1.42 & +0.17 & 6{,}874 \\
Llama 3.1 405B & Adjusted             & 0.456 & 0.048 & 1.80 & +0.28 & 28{,}283 \\
GLM-4.5 & Tie                 & 0.649 & 0.072 & 1.47 & +0.19 & 6{,}874 \\
GLM-4.5 & Adjusted             & 0.489 & 0.048 & 1.83 & +0.29 & 28{,}283 \\
\bottomrule
\end{tabular}
\label{tab:mc-alpha}
\end{table}

% DC: Reminders to self post-submission
% 1. Add details for remaining two experiment (broader table below)
% 2. Attach figures for the above - I've seen them and they look insanely compelling


% Here we provide the per-model results with 95\% confidence intervals from the \emph{Reward Modelling Sanity Check} analysis.

% \begin{table}[h]
% \centering
% \small
% \caption{OpenAI Summarize-from-Feedback. Agreement rates with 95\% CIs.}
% \label{tab:permodel-summarize}
% \begin{tabular}{@{}lcc@{}}
% \toprule
% Model & Agreement (\%) & 95\% CI \\
% \midrule
% Gemma-3-4B & 52.4 & [50.4, 54.3] \\
% Qwen-3-4B-Base & 51.6 & [49.7, 53.6] \\
% Gemma-3-27B & 54.3 & [52.3, 56.2] \\
% Llama-3.1-8B & 54.2 & [52.2, 56.1] \\
% Llama-3.1-70B & 56.4 & [54.5, 58.3] \\
% \bottomrule
% \end{tabular}
% \end{table}

% \begin{table}[h]
% \centering
% \small
% \caption{UltraFeedback (binarized). Agreement rates with 95\% CIs.}
% \label{tab:permodel-ultra}
% \begin{tabular}{@{}lcc@{}}
% \toprule
% Model & Agreement (\%) & 95\% CI \\
% \midrule
% Gemma-3-4B & 57.5 & [55.3, 59.6] \\
% Qwen-3-4B-Base & 59.0 & [56.8, 61.1] \\
% Gemma-3-27B & 60.2 & [58.0, 62.3] \\
% Llama-3.1-8B & 58.0 & [55.8, 60.1] \\
% Llama-3.1-70B & 59.5 & [57.3, 61.6] \\
% \bottomrule
% \end{tabular}
% \end{table}

% \begin{table}[h]
% \centering
% \small
% \caption{NVIDIA HelpSteer. Agreement rates with 95\% CIs.}
% \label{tab:permodel-helpsteer}
% \begin{tabular}{@{}lcc@{}}
% \toprule
% Model & Agreement (\%) & 95\% CI \\
% \midrule
% Gemma-3-4B & 57.8 & [56.2, 60.2] \\
% Qwen-3-4B-Base & 60.8 & [58.9, 62.8] \\
% Gemma-3-27B & 58.4 & [55.4, 60.7] \\
% Llama-3.1-8B & 56.2 & [53.9, 59.7] \\
% Llama-3.1-70B & 59.8 & [58.6, 62.0] \\
% \bottomrule
% \end{tabular}
% \end{table}

% \begin{table}[h]
% \centering
% \small
% \caption{Skywork Reward Preference. Agreement rates with 95\% CIs.}
% \label{tab:permodel-skywork}
% \begin{tabular}{@{}lcc@{}}
% \toprule
% Model & Agreement (\%) & 95\% CI \\
% \midrule
% Gemma-3-4B & 59.6 & [57.6, 61.5] \\
% Qwen-3-4B-Base & 61.7 & [59.8, 63.6] \\
% Gemma-3-27B & 59.6 & [57.6, 61.5] \\
% Llama-3.1-8B & 58.8 & [56.9, 60.8] \\
% Llama-3.1-70B & 59.6 & [57.7, 61.6] \\
% \bottomrule
% \end{tabular}
% \end{table}



\subsection{Typicality Bias in More Preference Datasets}\label{appendix:preference_bias_base_model}

  %Besides the causal analysis, w
  We also investigate whether typicality bias exists in more preference datasets and base models. We evaluate four widely-used preference datasets on five representation base models (Gemma-3-4B, Qwen3-4B,
  Gemma-3-27B, Llama-3.1-8B, Llama-3.1-70B). The preference
  datasets span different domains and annotation methodologies: OpenAI TL;DR~\citep{stienon2020learning} (human-annotated
  summarization), UltraFeedback~\citep{cui2023ultrafeedback} (GPT-4 annotations), NVIDIA HelpSteer-v2~\citep{wang2024helpsteer2} (human ratings), and
  Skywork Preference~\citep{liu2024skywork} (hybrid).

  \textbf{Experimental Setup.} As most of these datasedo not have separate labels for correctness and helpfulness, it is infeasible to apply the Bradley-Terry logistic model as before. Instead, for each preference dataset, we calculate the typicality bias rate, which measures how often the human-preferred response in a preference pair is assigned a higher likelihood by a base model. We sample 2,500 preference
   pairs from each dataset and compute the typicality bias ratio with 95\% confidence intervals. 

  \textbf{Results.} The results are shown in \Cref{fig:cognitive_bias_panels}. Our findings reveal the underlying typicality biases across all base
  models. Most critically, the typicality bias rate  consistently exceed the 50\% chance
  baseline by 4-12 percentage points, indicating that human annotators do exhibit preferences towards more typical texts  under various base models. Besides, larger models (e.g., Gemma-3-27B, Llama-3.1-70B) show higher typicality bias rates. %This suggests that typicality biases emerge
  %during pre-training from underlying data distributions and model architectures.

%   The bias patterns show remarkable consistency: larger models (Llama-3.1-70B) tend to exhibit
%   stronger preference alignment, while smaller models show more variability. These results have significant implications for preference learning: RLHF and other preference
%   optimization methods may amplify existing biases rather than learning preferences de novo,
% resulting in the main cause of mode collapse or reduced diversity in model outputs.

  \begin{figure}[!htbp]
      \centering
      \includegraphics[width=\linewidth]{figures/appendix/cognitive_bias_combined.pdf}
      \caption{\textbf{Typicality bias rate across different preference datasets and base models.} Typicality bias rate measures how often the human-preferred response in a preference pair is assigned a higher likelihood by a base model. 
      All models show a systematic, above-chance bias (agreement >50\%), with larger models generally exhibiting a stronger effect.
      We also show the 95\% confidence intervals. 
      The consistent above-chance preference shows that there exists a \textit{typicality biases} in human preference data. 
      % \wyshi{Ask Derek whether to put in the main text}
  }
      \label{fig:cognitive_bias_panels}
  \end{figure}
  
\subsection{How Typicality Bias Causes Mode Collapse}
\label{app:power-sharpening}

\cite{rafailov2024directpreferenceoptimizationlanguage} shows that the closed-form solution to the KL-regularized RLHF objective in \eqref{eq:objective} is the following:
\begin{equation}
\pi^*(y\mid x) = \frac{1}{Z(x)}\,\pi_{\mathrm{ref}}(y\mid x)\,\exp\!\left(\frac{r(x,y)}{\beta}\right)
\end{equation}

Substituting our reward decomposition from \eqref{eq:bt-assumption}, we have:
\begin{align}
\pi^*(y\mid x) &= \frac{1}{Z(x)}\,\pi_{\mathrm{ref}}(y\mid x)\,\exp\!\left(\frac{r_{\text{true}}(x,y) + \alpha\,\log \pi_{\mathrm{ref}}(y\mid x) + \epsilon(x)}{\beta}\right) \nonumber\\
&= \frac{\exp(\epsilon(x)/\beta)}{Z(x)}\,\pi_{\mathrm{ref}}(y\mid x)^{1+\alpha/\beta}\,\exp\!\left(\frac{r_{\text{true}}(x,y)}{\beta}\right)
\end{align}

Since the partition function $Z(x)$ contains the same $\exp(\epsilon(x)/\beta)$ factor, this cancels, yielding:
\begin{equation}
\pi^*(y\mid x) \propto \pi_{\mathrm{ref}}(y\mid x)^{\gamma}\,\exp\!\left(\frac{r_{\text{true}}(x,y)}{\beta}\right), \quad \gamma := 1 + \frac{\alpha}{\beta}
\label{eq:power-result}
\end{equation}

This power transform with exponent $\gamma > 1$ (when $\alpha > 0$) sharpens the reference distribution, amplifying its modes while suppressing the tails. The effect strengthens as the typicality bias $\alpha$ increases or the KL penalty $\beta$ decreases. In the limiting case where true task utility is approximately flat over a set $\mathcal{S}$, the optimal policy reduces to $\pi^*(\cdot \mid x) \propto \pi_{\mathrm{ref}}(\cdot \mid x)^{\gamma}$ on $\mathcal{S}$, producing mode collapse toward the most typical responses under $\pi_{\mathrm{ref}}$.



% claims about prompting section is here:
\subsection{Different Prompts Collapse to Different Modes: An Analysis of Prompt Capability Under Mode Collapse}
\label{appendix:anthony's proof on the mode for different prompts}
\paragraph{Setup.} For a fixed prompt $x_\text{orig}$, we are interested in recovering the full diversity inherent to the reference policy $\pi_\text{ref}(\cdot | x_\text{orig})$. We hope to do so for some corresponding affected set $\mathcal{S}_{\text{orig}}$, where $\pi^*$ is mode collapsed. Specifically, mode collapse means:
\begin{equation}
    \pi^*(y | x) = \delta_{y^*}(y) \quad\text{on } \ \mathcal{S}_\text{orig}, \quad\text{where}\quad y^* \in \text{arg}\max\nolimits_y \pi_\text{ref}(y | x)
\end{equation}
and $\delta$ is the Dirac function: $\delta_{y^*}(y) = \{1 \text{ if }y^* = y, 0\text{ else}\}.$

To recover diversity, we assume a new prompt $x$, which is possibly distinct from $x_\text{orig}$, and a (new) sampling strategy that may extend beyond direct sampling of the policy $\pi^*(\cdot | x)$. Since we demonstrated the potential for mode collapse of $\pi^*$ independent of prompt, we also assume $\pi^*(\cdot | x)$ remains mode collapsed on some set $\mathcal{S}$.
\paragraph{A Stronger Notion of Mode Collapse for $x$.} For tractability, we assume $\pi^*(\cdot | x)$ is mode collapsed on all of $\mathcal{Y}$ ($\mathcal{S} = \mathcal{Y}$). While coarse, this assumption is justified in practice: repeated samples from $\pi^*$ return the same completion with high probability, implying that the total probability mass away from this completion (the mode $y^*$) is negligible. From the perspective of observable sampling behavior, $\pi^*$ is effectively mode collapsed on all of $\mathcal{Y}$; it is mode collapsed to $y^*$ on some set and has near-zero probability everywhere else.

\paragraph{Specifying Sampling Procedures.} To compare probabilities between different prompts of $\pi^*$ and $\pi_{\text{ref}}$, we need to account for how a single completion is chosen from the result of each prompt. This process defines a completion's new (non-mode-collapsed) probability under the prompt.
\begin{enumerate}
    \item Instance-level prompts (the standard case) return only one completion. Here, we can directly compare the probability assigned by $\pi^*$ and $\pi$.
    \item List-level prompts return several possible completions, but no probabilities. The natural assumption, without added information, is that each completion is chosen at random with equal probability.
    \item Distribution-level prompts return completions together with probability estimates. In this case, it is reasonable to assume that sampling follows the provided probabilities.
\end{enumerate}
This distinction explains why distribution-level prompts can accurately replicate $\pi_{\text{ref}}$, as we prove next. It also aligns with our experimental results comparing $\pi^*$ under distribution-level prompting with $\pi_{\text{ref}}$ in \S\ref{appendix:probing_pre_training_data}.

\paragraph{Claim 1} \textit{Instance-level prompts return the mode of $\pi_\text{ref}$.} 
\begin{proof}
Let $x = x_\text{orig}$. Since $\pi^*(\cdot | x)$ is collapsed, we know $\pi^*(y | x) = \delta_{y^*}(y)$ for any $y$. So, all probability is on the mode of $\pi_\textrm{ref}(\cdot |x)$. Any sample $y \sim \pi^*(y | x)$ returns this mode almost surely.
\end{proof}

\paragraph{Claim 2} \textit{List-level prompts return uniform distributions at best.}
\begin{proof}
Fix the list prompt $x \neq x_\text{orig}$ and let $Z \sim \pi^*(\cdot | x)$ be the random completion for this list prompt (presumably, a list of completions itself). To process lists, assume a list parser $\phi : \mathcal{Y} \to \mathcal{Y}^*$ and write $\phi(Z) = \{Y_i\}_{i=1}^k$. Then, by the rule of total probability, the probability of any completion $y \in \mathcal{Y}$ is written
\begin{equation}
    \mathbb{P}(Y = y) = \sum_{z \in \mathcal{Y}} \ \mathbb{P}(Y = y | Z = z)\mathbb{P}(Z = z).
\end{equation}
Since $\pi^*$ is mode collapsed, $\mathbb{P}(Z = z) = \pi^*(z | x) = \delta_{y^*}(z)$ for all $z$. Thus, because $\delta_{y^*}(z)$ is null for all $z \neq y^*$, the probability simplifies:
\begin{equation}
    \mathbb{P}(Y = y) = \mathbb{P}(Y = y | Z = y^*) = \frac{1}{|\phi(y^*)|} \sum_{y_i \in \phi(y^*)}  \delta_{y_i}(y),
\end{equation}
where the last part leverages the fact that we sample from list-level prompts uniformly at random. When $\phi(y^*)$ is a list of distinct elements -- as requested in the list-level prompt -- this simplifies further:
\begin{equation}
    \mathbb{P}(Y = y) = \mathbb{P}(Y = y | Z = y^*) = \frac{1}{|\phi(y^*)|}.
\end{equation}
This is true because $y = y_i$ can only hold a single element of the (distinct) list $\phi(y^*)$. So, we recover a uniform distribution over the elements of $\phi(y^*)$.
\end{proof}

\paragraph{Claim 3} \textit{Distribution-level prompts can approximate $\pi_\text{ref}(\cdot | x_\text{orig})$.}
\begin{proof}
Fix a distribution prompt $x \neq x_\text{orig}$ and let $Z \sim \pi^*(\cdot | x)$ be the random completion for this distribution prompt (presumably, a list of completions itself with associated probabilities). To process, assume a parser $\phi : \mathcal{Y} \to \mathcal{Y}^k \times \Delta(k)$ where $\Delta(k)$ is the probability simplex on $k$ elements. Write $\phi(Z) = \{(Y_i, P_i)\}_{i=1}^k$ for the parsed completion $Z$. As before, by the chain rule of probability, the probability of any completion $y \in \mathcal{Y}$ is written
\begin{equation}
    \mathbb{P}(Y = y) = \sum_{z \in \mathcal{Y}} \ \mathbb{P}(Y = y | Z = z)\mathbb{P}(Z = z).
\end{equation}
As in \textbf{Claim 2}, this simplifies, owed to mode collapse of $\pi^*$:
\begin{equation}
    \mathbb{P}(Y = y) = \mathbb{P}(Y = y | Z = y^*) = \sum_{(y_i,p_i) \in \phi(y^*)}  p_i\delta_{y_i}(y).
\end{equation}
Different from \textbf{Claim 2}, the last part leverages the fact that we sample from distribution-level prompts according to the values $(p_i)_i$. This is an intuitive result: $P(Y = y) = p_i$ for each $y_i$ in the sequence returned by $\pi^*(\cdot | x)$.

The final goal is to see how $\mathbb{P}(Y = y)$ can replicate $\pi_\textrm{ref}(\cdot | x_\text{orig})$. We provide a constructive argument. Start by indexing each unique element $y \in \mathcal{Y}$, resulting in a sequence $(y_i)_{i=1}^m$ for $m = |\mathcal{Y}|$\footnote{It is reasonable to assume $\mathcal{Y}$ is finite because all computer representations are necessarily finite due to fixed memory. More practically speaking, we typically assume completions to be finite combinations of a finite token alphabet, which implies $\mathcal{Y}$ is finite.} where $y_i \neq y_j$ for $i \neq j$. This index enforces that $\delta_{y_i}(y)$ returns 1 for a single unique $y$. Then, we have:
\begin{equation}
    \forall i \in [m] \ : \ \pi_\text{ref}(y_i | x_\text{orig}) = \pi_\text{ref}(y_i | x_\text{orig}) \delta_{y_i}(y_i) + \underbrace{\sum_{j \neq i} \pi_\text{ref}(y_j | x) \delta_{y_i}(y_j)}_{j \neq i \ \Rightarrow \ \sum \ = \ 0} = \pi_\text{ref}(y_i | x_\text{orig}).
\end{equation}
Leveraging this equality, we can write $\pi_\text{ref}(\cdot | x')$ as below:
\begin{equation}
    \pi_\text{ref}(y | x_\text{orig}) = \sum_{i=1}^m \pi_\text{ref}(y_i | x_\text{orig}) \delta_{y_i}(y).
\end{equation}
Immediately, we see how distribution-level prompts can encode $\pi_\text{ref}(y | x_\text{orig})$. Specifically, we can set $p_i = \pi_\text{ref}(y_i | x_\text{orig})$ and $k=m$, assuming a shared index between $\phi(Z)$ and $\mathcal{Y}$. Then,
\begin{equation}
    \mathbb{P}(Y = y) = \sum_{(y_i,p_i) \in \phi(y^*)}  p_i\delta_{y_i}(y) = \sum_{i=1}^m p_i \delta_{y_i}(y) = \sum_{i=1}^m \pi_\text{ref}(y | x_\text{orig})\delta_{y_i}(y).
\end{equation}
In the last summand, $\delta_{y_i}(y)$ returns 1 only when $y = y_i$, so we have
\begin{equation}
    \mathbb{P}(Y = y) = \pi_\text{ref}(y | x_\text{orig}).
\end{equation}
\end{proof}
\paragraph{Remark.} An important part of the argument for \textbf{Claim 3} was our choice of the probabilities $p_i$, which implicitly means we are choosing the quality of $\pi^*$ in our construction -- text sampled from $\pi^*$ must be sufficiently accurate to encode distributional information, from $\pi_\text{ref}$, about the elements of $\mathcal{S}_\text{orig}$. In practice, we expect to observe some error here; e.g.,
\begin{equation}
    \forall i \in [m] \ : \ |p_i -  \pi_\text{ref}(y_i | x_\text{orig})| \leq \varepsilon.
\end{equation}
In this case, one can still directly show that 
\begin{equation}
    |\mathbb{P}(Y = y) - \pi_\text{ref}(y | x_\text{orig})| \leq \varepsilon
\end{equation}
as well by following a nearly identical argument. The takeaway is: although we make a strong assumption in our construction (i.e., perfect modeling of $\pi_\text{ref}$) this result also holds for subpar policies $\pi^*$ with proportional bounds on error. For example, see our experimental results in \S\ref{appendix:probing_pre_training_data}. In theory, since list-level prompts always return a uniform distribution, they do not share this property.

% OLD MAIN TEXT VERSION IS BELOW:
% Given a policy $\pi$, we consider a prompt $x$ with an associated sampling strategy $s_x$, as defined below. %Consider the following definitions of $x$ and $s_x$.
% % Here, we allow $s_x$ to sample from $\pi$ in non-traditional ways. For example, consider the following definitions of $x$ and $s_x$.
% \begin{enumerate}[nolistsep, leftmargin=*]
% \item \textbf{Instance-level prompts}: The prompt $x$ is the most traditional one, requesting one instance (e.g., ``Tell a joke about coffee''). So $s_x$ acts like an identity function: $s_x(\pi(\cdot | x)) = \pi(\cdot | x)$. 
% %In contrast to these non-traditional sampling strategies, we can also sample $\pi$ in the usual manner. So, $s_x$ would act like an identity function: $s_x(\pi(\cdot | x)) = \pi(\cdot | x)$.
%     \item \textbf{List-level prompts}: The prompt $x$ requests a list of outputs (e.g., ``Tell me $k$ jokes''), as used in~\cite{zhang_diverging_2024, meister_benchmarking_2024}. Without further information on this list, a reasonable choice for $s_x$ is to %sample $\pi(\cdot | x)$ to produce a list and then 
%     randomly sample any element from it with uniform probability as the output. %\as{potentially useful to provide a pointer to other work here}
%     \item \textbf{Distribution-level prompts}: The prompt $x$ requests $k$ outputs with corresponding probabilities. For example, for the prompt ``Tell $k$ jokes about coffee with their probabilities'', an instruction-tuned model generates $k$ distinct jokes $J_i$ with $k$ probabilities $p_i$. $s_x$ can randomly pick from each $J_i$ according to the associated probabilities $p_i$.
%     % (DC: Actively working on this section...)
% \end{enumerate}
% In general, we use $s_x(\pi(\cdot | x))$ to denote the distribution defined by sampling according to $s_x$.
% %We consider prompts that admit a dual sampling process \wyshi{why dual? what's Q?}: $Q \sim \pi^*(\cdot | x)$, then $y \sim Q$. For example, consider the \textbf{distribution-level} prompt: “Tell me $k$ jokes about coffee with their probabilities.” An instruction-tuned model generates $k$ distinct jokes $J_i$ with $k$ probabilities $p_i$. Then, we can sample from the induced distribution $Q = \sum_{i=1}^k p_i\delta_{J_i}$. \wyshi{what's $\delta$ here}

% Intuitively, we want to find a prompt $x$ and a sampling strategy $s_x$ that can %where it would be feasible to 
% recover the diverse base model $\pi_\text{ref}$, even when $\pi$ is mode collapsed. The best thing we could do is to use that $x$ and $s_x$ to train a model $\pi$ to replicate $\pi_\text{ref}$. \wyshi{i am here}

% % We consider this recovery feasible if we can successfully use that $x$ and $s_x$ to train a model $\pi$ to replicate $\pi_\text{ref}$.

% Formally, we define feasibility by whether it would be possible to train a model $\pi$ that can replicate $\pi_\text{ref}$ when using $x$ and $s_x$. Assuming a standard cross-entropy loss on data generated by $\pi_\text{ref}$, this training procedure is encoded in the following:
% \begin{equation}
% M(x)\;=\;\arg\min\nolimits_{\pi} \ \mathbb{E}_{y\sim \pi_{\textrm{ref}}(\cdot | x')} \ \!\big[-\log s_x(\pi(y|x))\big].
% % \;=\;\arg\min_{q\in\mathcal A(x)} \KL\!\big(p\ \|\ q\big)
% % }
% \label{eq:i-proj}
% \end{equation}
% $M(x)$ is the policy that minimizes loss for the prompt pair $(x, s_x)$. It is the best we can possibly do at replicating $\pi_\text{ref}$ if we restrict ourselves to the prompt pair $(x, s_x)$.
% % With this notation, a prompt is a constraint on the types of distributions a model may report. For a given prompt $x$, write $\mathcal A(x)\subseteq\simplex(Y)$  \wyshi{what's $\mathcal A(x)$ and $\Delta(Y)$} \as{TODO, add some short explanation,} 
% % for the set of \emph{admissible reports} \wyshi{admissible formats?} (e.g., “point answer,” “top‑$k$ with weights,” “full distribution”). Among the distributions a prompt admits, the maximum likelihood estimate according to data from $\pi_\textrm{ref}(\cdot | x)$ is given below \wyshi{not following the equation} \wyshi{pick q to distill to minimize, a prompt that allows you to pick a model that fits pi ref}
% % \wyshi{a(x) = roll a dice, a(x) = give a distribution of dice rolling, } 
% % \as{TODO, Derek}
% % :
% % \begin{equation}
% % M(x)\;=\;\arg\max\nolimits_{Q \in \mathcal{A}(x)} \ \mathbb{E}_{y\sim \pi_{\textrm{ref}}(\cdot | x')} \ \!\big[\log Q(y)\big].
% % % \;=\;\arg\min_{q\in\mathcal A(x)} \KL\!\big(p\ \|\ q\big)
% % % }
% % \label{eq:i-proj}
% % \end{equation}
% % Ideally, a prompt admits distributions such that $M(x) = \pi_\textrm{ref}(\cdot | x)$. In this case, it would be possible for some high-quality trained policy $\pi^*$ to recover $\pi_\textrm{ref}(\cdot | x)$ via prompting alone -- whether $\pi^*$ is mode collapsed or not.
% %%% OLD version
% % We view a \emph{prompt as a constraint on what the model may report}.
% % Let $Y$ be a finite outcome space and let $p(\cdot\mid x)$ denote the model’s latent predictive distribution for prompt $x$. Fix a single strictly proper scoring rule—the \emph{log score} $S(q,y)=\log q(y)$. For a given prompt $x$, write $\mathcal A(x)\subseteq\simplex(Y)$ for the set of \emph{admissible reports} (e.g., “point answer,” “top‑$k$ with weights,” “full distribution”). The log‑loss Bayes act is the information projection of $p$ onto $\mathcal A(x)$:
% % \as{Need to introduce proper scoring rules (quickly) and make clear why we look at them}.
% % \as{We can drop unnecessary terminology like i-projection}.
% % \derek{Agreed: What other terms can we take out?}
% % \as{Why expectation wrt p?} 
% % % DC: Possible useful thing to note: "Other strictly proper rules are possible; we focus on log for concreteness and its link to KL"
% % \begin{equation}
% % \boxed{
% % M(x)\;=\;\arg\max_{q\in\mathcal A(x)}\ \E_{Y\sim p}\!\big[\log q(Y)\big]
% % \;=\;\arg\min_{q\in\mathcal A(x)} \KL\!\big(p\ \|\ q\big)
% % }
% % \label{eq:i-proj}
% % \end{equation}
% % since $\E_p[\log q]= -\KL(p\|q)-H(p)$ and $H(p)$ is constant in $q$ (see Appendix~\ref{app:iproof} for complete proof).

% Three immediate consequences are useful to observe. Proof is provided for all of these claims in Appendix~\ref{}. \wyshi{how to prove these? do i just plug in the $s_x$ up there?}
% \begin{enumerate}[leftmargin=*, nolistsep]
% \item \textbf{Instance-level prompts return a mode.}
% When $\pi$ is mode collapsed (i.e., \textit{sharpened}), it can be shown that instance-level prompts can only return the mode of $\pi_\textrm{ref}$: $M(x) = \delta_{y^*}$ with $y^*\in\arg\max_{y\in Y}\pi_\text{ref}(y\mid x)$. We show this by assuming $\pi$ puts most of its probability on one outputs and a small constant amount on all other outputs. The key takeaway is that we fail to recover any of the diversity expressed by $\pi_\text{ref}$.
% % Assume only \emph{point answers} are admissible and $\pi^*(\cdot | x)$ is mode collapsed (i.e., \textit{sharpened}). Then, we can model $\mathcal{A}(x)$ as a family of smoothed Dirac $\delta$ functions \wyshi{what's this, and why Dirac $\delta$?}: one completion holds most of the probability under $\pi^*$ and the rest hold a small amount (constant for convenience). In this case, it can be shown that $M(x) = \delta_{y^*}$ with $y^*\in\arg\max_{y\in Y}\pi_\text{ref}(y\mid x)$. We fail to recover all of $\pi_\text{ref}(\cdot \mid x)$. In fact, we only recover its mode.
% \item \textbf{List-level prompts also suffer.} Similar to instance-level prompts, list-level prompts also fail to recover the full distribution $\pi_\text{ref}(\cdot \mid x)$ when $\pi$ is mode collapsed. Instead, $M(x)$ would be a uniform distribution \wyshi{why uniform} \wyshi{may add a bit more intuitive rational, the only way would be uniform, the way to maximize diversity is uniform
% } over the $k$ most likely outputs according to $\pi_\text{ref}(\cdot \mid x)$. \as{I think specifying what we mean by list-level before solves this? lmk}
% %$\mathcal A_{\text{point}}(x)=\{\delta_y:\,y\in Y\}$. Interpreting \eqref{eq:i-proj} on the closure of admissible reports (Diracs as limits of smooth distributions), the projection selects any $y^*\in\arg\max_{y\in Y}p(y\mid x)$. A short limit argument appears in Appendix~\ref{app:point-limit}. \as{if q is a point mass, then DKL is inf when p not absolutely continuous to q and 0 iff q = p. so p would need also be a point mass (I think) for this to be true. We may need to find a workaround for this part of proof. Maybe since we only care about mode collapsed p anyway we can just assume it’s a point mass }
% % \as{TODO: possibly switch to simpler variance argument in slack to avoid this limit proof.}
% % \derek{WIP: I had 2-3 other ways of getting to this, this was the cleanest notationally and I think there was a trick needed - more to come}
% \item \textbf{Distribution prompts recover $\pi_\text{ref}$.} Only distribution-level prompts have the capability to recover $\pi_\text{ref}(\cdot \mid x)$ if $\pi$ is mode collapsed. 
% % If $\mathcal A(x) = \simplex(Y)$, then the unique minimizer is $M(x) = \pi_\textrm{ref}(\cdot\mid x)$ because $\log$ is a strictly proper score \wyshi{what's proper score} \citep{gneiting2007strictly}.
% If $k$ is large enough, then $s_x(\pi(\cdot | x)$ can approximate \textit{any} distribution over outputs. Specifically, if $\pi^*$ is optimal among mode collapsed policies, then $M(x)$ is the unique optimum and $s_x(\pi^*(\cdot | x)) \to \pi_\text{ref}$ exactly as $k$ grows. In practice, $k$ does not need to be very large to observe benefits. We use $k$ ranging from $5$ to $20$ in our experiments and conduct an ablation on $k$ in \cref{fig:num_candidates_ablation}\simoncomment{in xxx}.
% % \wyshi{does it matter we only use k=5 in some experiments?} \as{We can point to ablation study here? feel free to edit that blurb}
% %Distribution-level prompts allow $\mathcal{A}(x) \to \simplex(Y)$ with larger and larger $k$ \wyshi{does it matter we only use k=5 in some experiments?} -- that is, whenever $\pi^*$ is of sufficient quality to model $\pi_\text{ref}$ at request.
% \end{enumerate}
% \as{Derek, these could use discussed proofs in the appendix. The list-level claim is similar to the point-level claim.}

% \noindent
% Thus, Equation~\ref{eq:i-proj} is the key design lever: by changing $\mathcal A(x)$ with the prompt, we change what is optimal to report -- \emph{a single prototype} under point prompts versus \emph{the entire distribution} under distribution prompts. An optimal policy $\pi^*$, \textit{even one that is mode collapsed}, can still report $\pi_\text{ref}$ when requested via distribution-level prompting; here, other prompts fall provably short. This perspective connects directly to the typicality‑induced sharpening in Section~\ref{sec:typicality} and motivates our method below.%

% OLD COMMENTED OUT BELOW
% \subsection{Power Transform Sharpening}
% \label{app:power-sharpening}

% % --- Power-transform sharpening: derivation from (2) and (3) to (4) ---




% \paragraph*{Power-transform sharpening.} Plugging (3) into (2) yields the following step-by-step simplification.

% \medskip
% \noindent\textbf{What (2) and (3) say.}
% \begin{align*}
% \text{(2)}\qquad
% \pi^\star(y\mid x)
% &= \frac{1}{Z(x)}\;\pi_{\mathrm{ref}}(y\mid x)\,\exp\!\big(r(x,y)/\beta\big),
% \\[-2pt]
% Z(x) \;&=\; \sum_{y'} \pi_{\mathrm{ref}}(y'\mid x)\,\exp\!\big(r(x,y')/\beta\big)
% \quad\text{(replace $\sum$ by $\int$ if $y$ is continuous),}
% \\[6pt]
% \text{(3)}\qquad
% r(x,y) \;&=\; r_{\mathrm{sem}}(x,y) \;+\; \alpha\,\log \pi_{\mathrm{ref}}(y\mid x) \;+\; c(x),
% \qquad \alpha>0,
% \end{align*}
% where $c(x)$ does not depend on $y$.

% \medskip
% \noindent\textbf{Plug (3) into (2) and simplify.}
% \begin{align*}
% \pi^\star(y\mid x)
% &= \frac{1}{Z(x)}\;
% \pi_{\mathrm{ref}}(y\mid x)\,
% \exp\!\Big(\tfrac{r_{\mathrm{sem}}(x,y) + \alpha \log \pi_{\mathrm{ref}}(y\mid x) + c(x)}{\beta}\Big) \\[4pt]
% &= \frac{1}{Z(x)}\;
% \pi_{\mathrm{ref}}(y\mid x)\,\exp\!\big(r_{\mathrm{sem}}(x,y)/\beta\big)\,
% \exp\!\big(\alpha \log \pi_{\mathrm{ref}}(y\mid x)/\beta\big)\,
% \exp\!\big(c(x)/\beta\big) \\[4pt]
% &= \frac{\exp\!\big(c(x)/\beta\big)}{Z(x)}\;
% \big(\pi_{\mathrm{ref}}(y\mid x)\big)^{\,1+\alpha/\beta}\;
% \exp\!\big(r_{\mathrm{sem}}(x,y)/\beta\big),
% \end{align*}
% using $\exp(\alpha\log p/\beta)=p^{\alpha/\beta}$. The normalizer becomes
% \begin{align*}
% Z(x)
% &= \sum_{y'} \pi_{\mathrm{ref}}(y'\mid x)\,
% \exp\!\Big(\tfrac{r_{\mathrm{sem}}(x,y') + \alpha \log \pi_{\mathrm{ref}}(y'\mid x) + c(x)}{\beta}\Big) \\[2pt]
% &= \exp\!\big(c(x)/\beta\big)\;
% \sum_{y'} \big(\pi_{\mathrm{ref}}(y'\mid x)\big)^{\,1+\alpha/\beta}\;
% \exp\!\big(r_{\mathrm{sem}}(x,y')/\beta\big).
% \end{align*}
% The factor $\exp(c(x)/\beta)$ cancels, so defining
% \[
% \tilde{Z}(x) \;=\; \sum_{y'} \big(\pi_{\mathrm{ref}}(y'\mid x)\big)^{\,1+\alpha/\beta}\;
% \exp\!\big(r_{\mathrm{sem}}(x,y')/\beta\big),
% \]
% we obtain
% \[
% \boxed{\;
% \pi^\star(y\mid x)
% = \frac{\big(\pi_{\mathrm{ref}}(y\mid x)\big)^{\,1+\alpha/\beta}\;
% \exp\!\big(r_{\mathrm{sem}}(x,y)/\beta\big)}
% {\displaystyle \sum_{y'} \big(\pi_{\mathrm{ref}}(y'\mid x)\big)^{\,1+\alpha/\beta}\;
% \exp\!\big(r_{\mathrm{sem}}(x,y')/\beta\big)}
% \;\;\propto\;\;
% \big(\pi_{\mathrm{ref}}(y\mid x)\big)^{\,1+\alpha/\beta}\;
% \exp\!\big(r_{\mathrm{sem}}(x,y)/\beta\big)
% \;}
% \]
% which is exactly (4).

% % \medskip
% % \noindent\textbf{Quick sanity checks.}
% % \begin{itemize}
% % \item \emph{$c(x)$ disappears.} It contributes $\exp(c(x)/\beta)$ that cancels in normalization, as shown.
% % \item \emph{Sharpening exponent.} $\gamma = 1+\alpha/\beta>1$ (since $\alpha,\beta>0$), so $p \mapsto p^\gamma$ is a power transform that sharpens $\pi_{\mathrm{ref}}$.
% % \item \emph{Odds form (consistency).} For any $y_a,y_b$,
% % \[
% % \frac{\pi^\star(y_a\mid x)}{\pi^\star(y_b\mid x)}
% % = \Big(\frac{\pi_{\mathrm{ref}}(y_a\mid x)}{\pi_{\mathrm{ref}}(y_b\mid x)}\Big)^{1+\alpha/\beta}
% % \exp\!\Big(\frac{r_{\mathrm{sem}}(x,y_a)-r_{\mathrm{sem}}(x,y_b)}{\beta}\Big).
% % \]
% % \end{itemize}

% % \medskip
% % \noindent\textbf{Inline log-space version.} Sometimes it is convenient to write
% % \[
% % \log \pi^\star(y\mid x) \;=\; \Big(1+\frac{\alpha}{\beta}\Big)\log \pi_{\mathrm{ref}}(y\mid x)
% % \;+\; \frac{1}{\beta}r_{\mathrm{sem}}(x,y) \;-\; \log \tilde{Z}(x).
% % \]



% \subsection{Verbalized Sampling: Proof of the projection identity (Eq.~\eqref{eq:i-proj})}\label{app:iproof}

% We restate the claim and give a full proof, including existence/uniqueness conditions.

% \begin{proposition}[Projection identity under the log score]
% Let $Y$ be finite, let $p\in\Delta(Y)$ with support $S_p=\{y\in Y:\,p(y)>0\}$, and let $\mathcal A\subseteq \Delta(Y)$ be nonempty. Define $L(q)=\E_{Y\sim p}[\log q(Y)]$ with the convention $\log 0=-\infty$ and $0\log(0/q)=0$. Then the sets of optimizers coincide:
% \[
% \arg\max_{q\in\mathcal A} L(q)
% \;=\;
% \arg\min_{q\in\mathcal A} \KL\!\big(p\ \|\ q\big),
% \]
% where $\KL(p\|q)=\sum_{y\in Y} p(y)\log\!\big(p(y)/q(y)\big)$ takes value $+\infty$ whenever $q(y)=0$ for some $y\in S_p$.

% Moreover:
% \begin{enumerate}
% \item[(i)] (\emph{Existence}) If $\mathcal A$ contains at least one $q$ with $q(y)>0$ for all $y\in S_p$ and $\mathcal A$ is closed, then the minimum of $\KL(p\|q)$ over $\mathcal A$ is attained (and the same for the maximum of $L$).
% \item[(ii)] (\emph{Uniqueness}) If, in addition, $\mathcal A$ is convex and intersects the relative interior
% \[
% \mathrm{ri}\big\{q\in\Delta(Y): q(y)>0\ \forall\,y\in S_p\big\},
% \]
% then the minimizer is unique.
% \item[(iii)] (\emph{Distribution prompts}) If $\mathcal A=\Delta(Y)$, the unique optimizer is $q^\star=p$.
% \end{enumerate}
% \end{proposition}

% \begin{proof}
% \textbf{Equality of optimizer sets.}
% For any $q$ with $q(y)>0$ for all $y\in S_p$,
% \[
% L(q)=\sum_{y\in Y} p(y)\log q(y) \;=\; -\KL(p\|q)-H(p),
% \]
% where $H(p)=-\sum_y p(y)\log p(y)$ does not depend on $q$. If $q(y)=0$ for some $y\in S_p$, then $L(q)=-\infty$ and $\KL(p\|q)=+\infty$, so such $q$ cannot optimize either objective. Hence the $\arg\max$ of $L$ over $\mathcal A$ coincides with the $\arg\min$ of $\KL(p\|q)$ over $\mathcal A$ (as extended-valued optimizations).

% \textbf{(i) Existence.}
% Because $Y$ is finite, $\Delta(Y)$ is compact. The map $q\mapsto \KL(p\|q)$ is lower semicontinuous on $\Delta(Y)$ when extended with the value $+\infty$ on $\{q:\exists\,y\in S_p,\ q(y)=0\}$. If $\mathcal A$ is closed and contains some $q$ strictly positive on $S_p$, then $\mathcal A\cap\{q:\ q>0\ \text{on}\ S_p\}$ is nonempty and (being a closed subset of a compact set) is compact. Lower semicontinuity yields attainment of the minimum of $\KL(p\|q)$ on this set, hence also of the maximum of $L$.

% \textbf{(ii) Uniqueness.}
% On $\{q:\ q(y)>0\ \forall y\in S_p\}$,
% \[
% \KL(p\|q) = \text{const} \;-\; \sum_{y\in S_p} p(y)\log q(y),
% \]
% and $q\mapsto -\log q(y)$ is strictly convex on $(0,1)$. A positive weighted sum of strictly convex functions, restricted to an affine slice (the simplex), remains strictly convex; thus $\KL(p\|\cdot)$ is strictly convex on any convex set contained in $\{q:\ q>0\ \text{on}\ S_p\}$. If $\mathcal A$ is convex and intersects the relative interior of this domain, the minimizer of a strictly convex function over $\mathcal A$ is unique.

% \textbf{(iii) Case $\mathcal A=\Delta(Y)$.}
% Gibbs’ inequality gives $\KL(p\|q)\ge 0$ with equality iff $q=p$, so $p$ is the unique minimizer (and thus maximizer of $L$).
% \end{proof}

% \paragraph{Remarks.}
% (1) If $\mathcal A$ contains only distributions that place zero mass on some $y\in S_p$ (e.g., pure Dirac answers when $p$ is non-degenerate), then $\inf_{q\in\mathcal A}\KL(p\|q)=+\infty$ and $\sup_{q\in\mathcal A}L(q)=-\infty$; in practice we handle point prompts via soft-delta limits in Appendix~\ref{app:point-limit}. (2) For top-$k$ constraints with $k\ge |S_p|$, existence and uniqueness hold and the optimizer is $p$ restricted to the top-$k$ support and renormalized (the I-projection onto the $k$-sparse face).



% \subsection{Verbalized Sampling: Point prompts as limits}\label{app:point-limit}
% We justify the claim that \emph{point prompts return a mode} by treating point reports as limits of smooth distributions.

% \paragraph{Setup.}
% For each $y\in Y$ and temperature $\tau>0$, define a \emph{soft‑delta} $q_y^{(\tau)}\in\simplex(Y)$ such that $q_y^{(\tau)}(y)\to 1$ and $q_y^{(\tau)}(y')\to 0$ for $y'\neq y$ as $\tau\downarrow 0$. Two concrete families suffice:
% \begin{enumerate}
% \item \textbf{Sharpened categorical:} $q_y^{(\tau)}(y)=\frac{e^{1/\tau}}{e^{1/\tau}+ (|Y|-1)}$, $q_y^{(\tau)}(y')=\frac{1}{e^{1/\tau}+(|Y|-1)}$ for $y'\neq y$.
% \item \textbf{Mixture with uniform:} $q_y^{(\tau)}=(1-\tau)\,\delta_y+\tau\,u$, where $u$ is uniform on $Y$.
% \end{enumerate}
% Let $\mathcal A^{(\tau)}(x)=\{q_y^{(\tau)}:\,y\in Y\}$ so that $\overline{\mathcal A^{(\tau)}(x)}\to \mathcal A_{\mathrm{point}}(x)$ as $\tau\downarrow 0$.

% \paragraph{Claim.}
% For every fixed $\tau>0$, the minimizer of $q\mapsto \KL(p\|q)$ over $\mathcal A^{(\tau)}(x)$ is some $q_{y^*}^{(\tau)}$ with $y^*\in\arg\max_y p(y\mid x)$. Moreover, as $\tau\downarrow 0$, any sequence of minimizers $q_{y^*}^{(\tau)}$ converges (in distribution) to $\delta_{y^*}$ with $y^*\in\arg\max_y p(y\mid x)$.

% \paragraph{Proof sketch.}
% For any $q\in\mathcal A^{(\tau)}(x)$,
% \[
% \KL(p\|q) \;=\; -\sum_y p(y\mid x)\log q(y) - H(p).
% \]
% The $-H(p)$ term is constant, so minimizing $\KL$ is equivalent to maximizing $\sum_y p(y)\log q(y)$.
% For both constructions, $\log q_y^{(\tau)}(y)-\log q_y^{(\tau)}(y')$ is strictly increasing in $1/\tau$, and
% \[
% \Phi_{y}^{(\tau)} \;=\; p(y)\log q_y^{(\tau)}(y) + \sum_{y'\neq y} p(y')\log q_y^{(\tau)}(y')
% \]
% is strictly increasing in $p(y)$ because $q_y^{(\tau)}(y')$ is the same for all $y'\neq y$. Hence any maximizer must choose $y^*\in\arg\max_y p(y)$. Finally, epi/$\Gamma$‑convergence of $q\mapsto \KL(p\|q)$ on $\mathcal A^{(\tau)}(x)$ to its lower semicontinuous extension on $\mathcal A_{\mathrm{point}}(x)$ ensures $q_{y^*}^{(\tau)}\Rightarrow \delta_{y^*}$, establishing that \emph{the point‑prompt projection selects a mode}. \qedhere