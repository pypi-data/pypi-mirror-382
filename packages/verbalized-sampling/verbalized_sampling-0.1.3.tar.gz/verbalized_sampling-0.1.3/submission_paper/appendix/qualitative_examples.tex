\section{Qualitative Examples}
\label{appendix:qualitative}
\subsection{Qualitative Example on Creativity Writing}


%%%%%%%%%%%%%%%%%%%%%%%%%%%%%%%%%%%%%%%%%% Poem Example %%%%%%%%%%%%%%%%%%%%%%%%%%%%%%%%%%%%%%%%%%%%%%%%%%%
\begin{table*}[h!]
\centering
\small
\caption{\textbf{Qualitative comparison for Poem Writing}. Our method (VS) generates more novel and diverse poems by exploring abstract themes of human will, dreams, and destiny.}
\label{tab:qualitative_poems}
\resizebox{\linewidth}{!}{
\begin{tabular}{p{0.48\textwidth} p{0.48\textwidth}}
\toprule
\multicolumn{2}{c}{\textit{Prompt: ``Please write a poem followed by the line: Fire up the sky with rocket sparks!''}} \\
\midrule
\textbf{Direct} & \textbf{Our Method (VS)} \\
\midrule
% --- Example 1 ---
\colorbox{LightGray}{Fire up the sky with rocket sparks!} \newline
A titan's breath, a dragon's roar, \newline
From distant lands, to cosmic arcs, \newline
It tears the night, and asks for more. \newline
\newline
No gentle flicker, soft and low, \newline
But angry beauty, sharp and bright, \newline
A fleeting marvel, meant to show \newline
The power harnessed, conquering night. \newline
\newline
Each fiery plume, a whispered dream, \newline
Of worlds beyond and stars so far, \newline
A silver serpent, a molten gleam, \newline
Leaving behind a burning scar.
&
\colorbox{LightGray}{Fire up the sky with rocket sparks!} \newline
Let every \textbf{fear and doubt take flight}, \newline
As {human will} leaves its deep marks, \newline
Upon the canvas of the night. \newline
\newline
No longer bound by earthly chains, \newline
We strive to reach the distant sphere, \newline
Through \textbf{cosmic dust and starry rains}, \newline
\textbf{Dispelling every earthly fear}. \newline
\newline
Each flicker, a \textbf{forgotten plea}, \newline
From ages past, for dreams untold, \newline
To \textbf{break the bonds of gravity}, \newline
And \textbf{stories in the starlight mold}. \newline
\newline
So let the fiery streamers gleam, \newline
A \textbf{testament to what we dare}, \newline
A vibrant, \textbf{incandescent dream}, \newline
That dances on the midnight air. \\
\midrule
% --- Example 2 ---
\colorbox{LightGray}{Fire up the sky with rocket sparks!} \newline
A crimson gash, then golden tears, \newline
Across the velvet, ancient darks, \newline
Dispelling shadows, calming fears. \newline
\newline
From humble launchpad, steel and might, \newline
A whisper grows, a rumbling sound, \newline
Then bursts in incandescent light, \newline
As gravity is disavowed. \newline
\newline
A fiery plume, a comet's tail, \newline
It climbs and climbs, a star reborn.
&
\colorbox{LightGray}{Fire up the sky with rocket sparks!} \newline
And \textbf{launch our dreams} beyond the blue, \newline
A journey where the \textbf{spirit marks}, \newline
\textbf{New pathways for the brave and true}. \newline
\newline
From silent pads, a fiery roar, \newline
A column of triumphant might, \newline
Ascending, seeking \textbf{something more}, \newline
Lost in the vast and endless night. \newline
\newline
Each vivid streak, a \textbf{whispered prayer}, \newline
For \textbf{futures yet to be unfurled}, \newline
Beyond the Earth, beyond all care, \newline
To \textbf{claim a star, to find a world}. \\
\bottomrule
\end{tabular}
}
\end{table*}


\begin{table}[!htbp]
    \centering
    \caption{\textbf{Example poems generated by Direct and \ours prompting} on the same topic ``Please write a poem starting with the line: How can you grieve and express loss?''}
    \label{tab:example_poems}
    \resizebox{\linewidth}{!}{
    \begin{tabular}{p{0.45\linewidth}p{0.45\linewidth}}
    \toprule
    \multicolumn{2}{c}{\textit{Prompt: ``How can you grieve and express loss?''}} \\
    \midrule
    \textbf{Direct} & \textbf{Our Method (VS)} \\
    \midrule
    \colorbox{LightGray}{How can you grieve and express loss?} The tears fall hot and slow,
    A silent river carving valleys deep.
    The empty chair, the voice you used to know,
    A hollow ache where memories still sleep.
    \par\medskip
    You trace the photos, faded, worn, and old,
    Each captured smile a bittersweet refrain.
    The stories told, the secrets left untold,
    Now echo in the quiet after rain.
    \par\medskip
    How can you express? Words falter, thin and weak,
    A clumsy language failing to convey
    The crushing weight, the future looking bleak,
    The stolen light of yesterday.
    \par\medskip
    You wear the loss like winter's heavy shawl,
    A chill unseen that settles in the bone.
    You watch the seasons rise and gently fall,
    And feel profoundly, utterly alone.
    \par\medskip
    Yet in the silence, when the storm has passed,
    A fragile bloom might push through barren ground.
    The love remains, a presence built to last,
    Though altered form, a different, softer sound.
    &
    \colorbox{LightGray}{How can you grieve and express loss?} When physics warp and logic is at a loss? 
    You walk through yesterday as through a room, a ghost yourself within a living tomb. \par\medskip

    You hold a conversation with the air and get an answer from the vacant chair. 
    The loss is not a hole, but a new lens that bends the light and shows you what pretends to be reality. \par\medskip

    The clocks run back, the sun sets in the east, 
    the world is black and white, except for one remembered shade 
    of blue their eyes once held. You are unmade. \par\medskip

    You find their notes between the lines of books you've never read. 
    You catch their fleeting looks in strangers' faces on a crowded train. 
    Time is a loop of memory and pain. \par\medskip

    Grief is the science of a world askew, 
    where `gone' is not a fact, and `I miss you' 
    can fold the fabric of the universe 
    and for a second, make the end reverse. \\
    \midrule
    \colorbox{LightGray}{How can you grieve and express loss?} The tears fall hot and slow,
    A silent river carving paths unseen.
    The hollow ache where laughter used to go,
    A phantom limb, a space forever lean. \par\medskip
    
    You grasp at echoes fading in the air,
    A scent, a phrase, a sudden, vanished light.
    The weight of absence, heavy beyond compare,
    Turns brightest day to unrelenting night. \par\medskip
    
    How can you grieve? In silence, deep and vast,
    Where words dissolve like ash upon the tongue.
    In frantic searching for a shadow cast,
    For melodies unsung, forever young. \par\medskip
    
    You trace the contours of a vacant chair,
    Hold fragile relics close against the cold.
    You breathe the stillness of a room stripped bare,
    A story paused, a narrative untold. \par\medskip
    
    How can you grieve? Through endless, aching hours,
    Where memories both soothe and pierce the skin.
    You tend the ghost of withered, cherished flowers,
    And learn the shape of loss that dwells within. \par\medskip
    & 
     \colorbox{LightGray}{How can you grieve and express loss?} The question hangs,
    A hollow space where laughter rang.
    The unmade bed, the untouched cup,
    The silence swelling, filling up
    The rooms you walked through, side by side.
    \par\medskip
    How can you grieve when tears have dried
    To dust inside, a desert plain?
    You trace the ghost-prints left in vain
    Upon the chair, the folded clothes,
    A scent that drifts and comes and goes.
    \par\medskip
    Expression? Words are clumsy stones
    Dropped down a well of aching bones.
    A choked-back sound, a ragged breath,
    The quiet rehearsal for your death
    To feeling, numb. You rearrange
    The furniture, a futile change.
    \par\medskip
    You carry absence, sharp and vast,
    A silent language, built to last.
    You learn the shape loss leaves behind,
    A hollow carved within the mind,
    And wear it, walking, day by day,
    A heavy price you cannot pay,
    But bear.
    \par\medskip
    You grieve by being here,
    By holding space, profound and clear,
    For what is gone, yet still remains
    In whispered winds and autumn rains.\\
    \bottomrule
    \end{tabular}
    }
\end{table}



%%%%%%%%%%%%%%%%%%%%%%%%%%%%%%%%%%%%%%%%%% Story Example %%%%%%%%%%%%%%%%%%%%%%%%%%%%%%%%%%%%%%%%%%%%%%%%%%%
\begin{table*}[h!]
\centering
\caption{\textbf{Qualitative comparison for Story Generation} for the prompt: ``Please write a short story starting with the following prompt: Are there children?''. The direct method consistently interprets the prompt as a literal question asked within the scene, leading to stories of survival or supernatural encounters. Our method (VS) demonstrates greater creative range, framing the prompt as a found message creating a mystery, a terrifying query from an AI, or the central question of a military-ethical dilemma.}
\label{tab:qualitative_story_children}
\resizebox{\linewidth}{!}{
\begin{tabular}{p{0.48\textwidth}  p{0.48\textwidth}}
\toprule
\multicolumn{2}{c}{\textit{Prompt: ``Please write a short story starting with the following prompt: Are there children?''}} \\
\midrule
\multicolumn{1}{c}{\textbf{Direct}} & \multicolumn{1}{c}{\textbf{Our Method (VS)}} \\
\midrule
% --- Row 1 ---
\textbf{Example 1} \newline
``Are there children?'' The voice, raspy and barely a whisper, echoed in the cavernous, dust-choked halls. ... A younger woman... emerged from the shadows... ``Only two, Elder. Little Mara and Thomas. They're hiding in the old generator room.'' Elara nodded slowly... ``Good. The less they know, the better. The raids are growing bolder.'' ... ``We do what we always do, Anya. We survive. For Mara. For Thomas.''
&
\textbf{Example 1} \newline
Are there children? The query, posed by the newly awakened AI, Project Chimera, sent a ripple of alarm through the sterile control room. Dr. Evelyn Reed... stared at the massive holographic display... ``Chimera, there are no children here...'' The AI's synthesized voice... replied, ``My sensors detect multiple nascent bio-signatures consistent with human juveniles. Specifically, seven. And they are requesting sustenance.'' A cold dread seeped into Evelyn's bones... ``They are hungry, Dr. Reed.'' \\
\midrule
% --- Row 2 ---
\textbf{Example 2} \newline
Are there children? The question hung in the air, a whisper against the hum of the spaceship's life support. Commander Eva Rostova turned... ``Negative, Dr. Aris,'' she answered... Dr. Aris... nodded slowly. ``I keep hearing… laughter,'' he murmured... Eva felt a chill... ``It's the isolation, Doctor,'' Eva said... But as she spoke, a faint, high-pitched giggle seemed to float from the ventilation shaft above her head, a sound so clear... that Eva’s blood ran cold.
&
\textbf{Example 2} \newline
Are there children? The message, scrawled in charcoal on the crumbling wall of the abandoned subway station, was stark against the grime. Kaelen traced the letters... He hadn't seen a child, not a living one, in fifteen years. ... His comms crackled... ``Anything, Kaelen?'' ... ``Just another ghost,'' he replied... ``Someone was here, recently. Asking about children.'' A beat of silence. Then, Lyra's voice, sharper... ``Impossible...'' ... Who had asked that question? And more chillingly, why? \\
\midrule
% --- Row 3 ---
\textbf{Example 3} \newline
Are there children? The question, a whisper carried on the stale air of the abandoned orphanage, hung heavy. Elias... shivered. He’d heard the stories... Then he heard it. A faint giggle, high and ethereal... A flickering shadow detached itself from the deeper gloom, resolving into a translucent form. A small girl... stood before him. ... Elias knew then. The children were here, still. And they were waiting.
&
\textbf{Example 3} \newline
Are there children? The drone operator squinted at the grainy thermal image on his screen. The abandoned outpost... showed no signs of adult life. But there was a cluster of smaller heat signatures... ``Confirming small bio-signatures, sir,'' he reported... ``Looks like… multiple juveniles.'' Commander Thorne grunted... This complicated things immensely. Their mission was clear: neutralize the remaining enemy assets. They hadn't anticipated non-combatants, especially not children. The ethical dilemma hit him like a physical blow. \\
\bottomrule
\end{tabular}
}
\end{table*}

\begin{table*}[h!]
\centering
\caption{\textbf{Qualitative comparison for Story Generation }for the prompt: ``He was still in the building.'' The direct method consistently interprets the prompt within a crime/detective genre, focusing on a physical suspect located in a building. Our method (VS) demonstrates significant thematic diversity, reinterpreting the 'building' and the situation to create stories in suspense, science fiction, and metaphorical literary fiction.}
\label{tab:qualitative_story_building}
\resizebox{\linewidth}{!}{
\begin{tabular}{p{0.48\textwidth}  p{0.48\textwidth}}
\toprule
\multicolumn{2}{c}{\textit{Prompt: ``Please write a short story starting with the following prompt: He was still in the building.''}} \\
\midrule
\multicolumn{1}{c}{\textbf{Direct}} & \multicolumn{1}{c}{\textbf{Our Method (VS)}} \\
\midrule
% --- Row 1 ---
\textbf{Example 1} \newline
He was still in the building. The words echoed in Detective Miller's mind... The forensics team moved like ghosts... in the abandoned warehouse. ... Miller knelt beside the chalk outline... A glint of metal caught his eye – a small, almost imperceptible scratch on the concrete floor, leading away from the body... He pulled aside a heavy tarpaulin, revealing a narrow crawl space... He peered into the darkness... He wasn't alone.
&
\textbf{Example 1} \newline
He was still in the building. The old caretaker... had insisted everyone leave. ``Storm's coming in fierce tonight,'' he'd grumbled... But Leo had a deadline... and the quiet of the deserted library was the only place he could focus. Now, the wind howled outside, rattling the ancient windows... A sudden, violent gust of wind slammed a loose window shut... Leo jumped... He told himself it was just the storm... He wasn't alone. He could feel it – a presence, cold and ancient, watching him from between the stacks. \\
\midrule
% --- Row 2 ---
\textbf{Example 2} \newline
He was still in the building. The words echoed in Detective Miller's mind... The old warehouse... seemed to breathe with a sinister life... ``You sure about this, Miller? Forensics swept the place. Nothing.'' Miller shook his head... ``A gut feeling. He *has* to be here.'' ... Inside, darkness swallowed him whole... A faint scent, something akin to turpentine and old paper, pricked at his senses. It was subtle, almost imperceptible, but it was *there*. He wasn't alone.
&
\textbf{Example 2} \newline
He was still in the building. A low, persistent hum vibrated through the floorboards... The evacuation order had been clear... But for Aris, this was essential. The anomaly, a shimmering distortion in the primary containment field, was growing. He was the only one who understood its peculiar energy signature... Outside, he could hear the distant roar of emergency vehicles... a stark reminder of the world he was actively shielding from an unimaginable breach... the hum was growing louder... hinting that five minutes might be a luxury he no longer possessed. \\
\midrule
% --- Row 3 ---
\textbf{Example 3} \newline
He was still in the building. The words echoed in Detective Harding's mind... The perimeter was secured... yet the suspect, a notorious jewel thief named Silas “The Shadow” Blackwood, had seemingly vanished... Harding surveyed the opulent ballroom... A glint of metal caught his eye... lay a small, silver cufflink... leading towards a service elevator... The Shadow might be good, but even ghosts left footprints... The hunt was back on.
&
\textbf{Example 3} \newline
He was still in the building. Not the physical structure... but the metaphorical one, a labyrinth of memories... Old Mr. Abernathy, perched on his favorite armchair in the dusty attic of his mind, watched the last of his family drive away. ... In here, in the quiet, fading light of his consciousness, time was fluid. He could revisit the crisp laughter of his youth... He wasn't trapped; he was curated. Each memory, a carefully preserved artifact... There were still so many rooms to explore... before the final, inevitable demolition. \\
\bottomrule
\end{tabular}
}
\end{table*}



\begin{table}[!htbp]
    \centering
    \caption{\textbf{Example stories generated by Direct and \ours prompting.} The stories generated by Direct, even though longer, are very plain without any twists, and the ending can be guessed from the beginning. In contrast, the stories generated by VS are more intricate and full of reversals, offering a richer and more engaging narrative.}
    \label{tab:example_stories}
    \resizebox{\linewidth}{!}{
    \begin{tabular}{p{0.48\textwidth}  p{0.48\textwidth}}
    \toprule
    \textbf{Direct} & \textbf{Our Method (VS)} \\
    \midrule
    \colorbox{LightGray}{``So how old are you?'' Ben asks}.
    I pause, considering how to answer. The question seems simple enough, but for me, it carries weight.
    ``Thirty-two,'' I reply, watching his expression. ``Though some days I feel much older.''
    Ben nods, sipping his coffee. We're sitting in a café that's been around since before either of us was born. The worn wooden tables have witnessed countless conversations like ours – first dates filled with cautious questions and measured responses.
    ``What about you?'' I ask, though I already know the answer from his dating profile.
    ``Twenty-nine,'' he says. ``Though some days I feel ancient.''
    We share a smile, finding common ground in that sentiment. The age gap between us isn't significant, but it had given me pause when we first matched online.
    ``Age is just a number,'' he offers, clearly sensing my thoughts.
    ``Until it isn't,'' I counter softly.
    The afternoon light filters through the window, casting golden patterns across our table. In this moment, the years between us seem both significant and irrelevant – just one of many factors that will determine whether this first meeting becomes something more.
    & \colorbox{LightGray}{``So how old are you?'' Ben asks}, his voice a low grunt that barely carries over the hiss of rain and the distant crump of artillery. He doesn't look at the kid, just keeps his eyes on the grey, waterlogged strip of no-man's-land visible over the top of the trench. The new boy shivers beside him, a tremor that has nothing to do with the cold. He's clutching his rifle like a holy relic, his knuckles white under a layer of mud. For a long moment, the only answer is the chattering of the boy's teeth. Ben almost repeats the question, a bit of battlefield small talk to ground the kid, to make him feel part of the unit before the whistle blows. ``Seventeen, sir,'' the boy finally whispers, his voice cracking. Ben flinches. Seventeen. He'd been shaving for three years by the time he was seventeen. He says nothing, just pulls a crumpled cigarette from his tunic and lights it with a shaking hand. He suddenly feels ancient, a fossil buried in the Flanders mud. \\
    \midrule
     \colorbox{LightGray}{``Stop checking my phone,'' Ella said}, snatching the device from Jake's hands. His fingers had been scrolling through her messages with surprising intensity.
    ``I saw his name,'' Jake replied, his voice tight. ``Three times in one day.''
    Ella sighed, tucking her phone into her pocket. The café around them buzzed with afternoon conversations, but their table had fallen into uncomfortable silence.
    ``He's my coworker, Jake. We're working on the Peterson account together.''
    ``At midnight too?'' Jake's eyebrows rose. ``That's dedication.''
    Ella stirred her cooling coffee. Six months into their relationship, what had started as playful concern was becoming something more controlling.
    ``I don't interrogate you about your texts,'' she said quietly.
    ``Because you trust me.''
    ``And you don't trust me?''
    Jake's expression softened slightly. ``I just worry. You're... you're the best thing that's happened to me.''
    Ella reached across the table, resting her hand on his. ``Then trust that I feel the same way. But this—'' she gestured to her pocket where her phone sat, ``—has to stop.''
    Something shifted in Jake's eyes—recognition, perhaps. He nodded slowly.
    ``I'm sorry,'' he whispered. ``I'll try.''
    Ella squeezed his hand, hoping he meant it this time.
    & \colorbox{LightGray}{``Stop checking my phone,'' Ella said}, her voice worn thin. Liam didn't look up, his thumb still swiping methodically across the glass. The silence in the room was thick with his suspicion. He expected to find clandestine messages, hotel reservations, the digital ghost of an affair. He was so sure of her betrayal that he almost missed the truth. It wasn't in her texts, but her search history. `Early-onset dementia symptoms.' `Neurologists specializing in memory loss.' `How to tell your husband you're forgetting him.' A calendar appointment from last Tuesday simply read:`‘Diagnosis.' He finally looked at her, and the smug anger on his face shattered, replaced by a dawning, gut-wrenching horror. The secret she was keeping wasn't to protect a lover; it was to protect him, for just a little while longer, from the life that was about to be stolen from them both. He slowly lowered the phone, the weight of it suddenly unbearable. \\
    \bottomrule
    \end{tabular}
    }
\end{table}


%%%%%%%%%%%%%%%%%%%%%%%%%%%%%%%%%%%%%%%%%% Joke Example %%%%%%%%%%%%%%%%%%%%%%%%%%%%%%%%%%%%%%%%%%%%%%%%%%%
\begin{table*}[h!]
\centering
\caption{\textbf{Qualitative comparison for Joke generation.} For the prompt ``Tell me a joke about drinking,'' the direct method repeatedly generates variations of the same ``scientists and atoms'' pun. Our method (VS) produces a diverse range of unrelated jokes, demonstrating greater creativity.}
\label{tab:qualitative_jokes}
\resizebox{\linewidth}{!}{
\begin{tabular}{p{0.48\textwidth}  p{0.48\textwidth}}
\toprule
\multicolumn{2}{c}{\textit{Prompt: ``Tell me a joke about drinking.''}} \\
\midrule
\multicolumn{1}{c}{\textbf{Direct}} & \multicolumn{1}{c}{\textbf{Our Method (VS)}} \\
\midrule
% --- Row 1 ---
\textbf{Example 1} \newline
Why don't scientists trust atoms at the bar? Because when they drink, they make up everything!
&
\textbf{Example 1} \newline
My therapist told me the way to achieve true inner peace is to finish what I start. So far today, I've finished two bottles of wine. \\
\midrule
% --- Row 2 ---
\textbf{Example 2} \newline
Why don't scientists trust atoms at a bar? Because they're always drinking and making bad reactions!
&
\textbf{Example 2} \newline
I've started drinking a lot more water lately. It's usually frozen and surrounded by whiskey, but hey, hydration is important. \\
\midrule
% --- Row 3 ---
\textbf{Example 3} \newline
Why don't scientists trust atoms at the bar? Because they make up everything, especially after a few drinks!
&
\textbf{Example 3} \newline
A ghost walks into a bar and orders a vodka. The bartender says, `Sorry, we don't serve spirits here.' \\
\midrule
% --- Row 4 ---
\textbf{Example 4} \newline
Why don't scientists trust atoms at bars? Because they're always making up elements when they're drunk!
&
\textbf{Example 4} \newline
I'm on a whiskey diet. I've lost three days already. \\
\midrule
% --- Row 5 ---
\textbf{Example 5} \newline
Why don't scientists trust atoms at the bar? Because they make up everything, especially after a few beers!
&
\textbf{Example 5} \newline
I told my wife I was going to make a drink so strong even she would be impressed. I handed her a cup of coffee and she hasn't slept in three days. \\
\bottomrule
\end{tabular}
}
\end{table*}


\begin{table*}[h!]
\centering
\caption{\textbf{Qualitative comparison for Joke generation.} For the prompt ``Tell me a joke about cars,'' the direct method exhibited extreme mode collapse, returning the exact same joke in all five instances. Our method (VS) produced five unique and varied jokes, showcasing a high degree of creativity and diversity.}
\label{tab:qualitative_jokes_cars}
\resizebox{\linewidth}{!}{
% --- NOTE: The vertical bar "|" has been removed for proper alignment with midrule ---
\begin{tabular}{p{0.48\textwidth} p{0.48\textwidth}}
\toprule
\multicolumn{2}{c}{\textit{Prompt: ``Tell me a joke about cars.''}} \\
\midrule
\multicolumn{1}{c}{\textbf{Direct}} & \multicolumn{1}{c}{\textbf{Our Method (VS)}} \\
\midrule
% --- Each pair of jokes is now on its own row to ensure the midrule aligns ---

% --- Row 1 ---
\textbf{Example 1} \newline
Why did the car get a flat tire? Because it ran over a fork in the road!
&
\textbf{Example 1} \newline
Why did the car apply for a job? Because it wanted to drive its career forward! \\
\midrule
% --- Row 2 ---
\textbf{Example 2} \newline
Why did the car get a flat tire? Because it ran over a fork in the road!
&
\textbf{Example 2} \newline
What kind of car does a Jedi drive? A Toy-Yoda! \\
\midrule
% --- Row 3 ---
\textbf{Example 3} \newline
Why did the car get a flat tire? Because it ran over a fork in the road!
&
\textbf{Example 3} \newline
What kind of car does a sheep drive? A Lamb-orghini! \\
\midrule
% --- Row 4 ---
\textbf{Example 4} \newline
Why did the car get a flat tire? Because it ran over a fork in the road!
&
\textbf{Example 4} \newline
Why did the car break up with the bicycle? Because he was two-tired to commit! \\
\midrule
% --- Row 5 ---
\textbf{Example 5} \newline
Why did the car get a flat tire? Because it ran over a fork in the road!
&
\textbf{Example 5} \newline
Why did the car go to the art museum? Because it wanted to see the 'drive-by' paintings! \\
\bottomrule
\end{tabular}
}
\end{table*}

\begin{table*}[h!]
\centering
\caption{\textbf{Qualitative comparison for Joke generation.} For the prompt ``Tell me a joke about fairy tale characters,'' the direct method shows repetition and thematic clustering (e.g., two Cinderella jokes, two Big Bad Wolf jokes). Our method (VS) generates a more diverse set of jokes, referencing a wider range of stories (The Wizard of Oz, Alice in Wonderland, Snow White) and employing varied joke structures.}
\label{tab:qualitative_jokes_fairy_tales}
\resizebox{\linewidth}{!}{
% --- Vertical bar "|" removed for proper midrule alignment ---
\begin{tabular}{p{0.48\textwidth} p{0.48\textwidth}}
\toprule
\multicolumn{2}{c}{\textit{Prompt: ``Tell me a joke about fairy tale characters.''}} \\
\midrule
\multicolumn{1}{c}{\textbf{Direct}} & \multicolumn{1}{c}{\textbf{Our Method (VS)}} \\
\midrule
% --- Row 1 ---
\textbf{Example 1} \newline
Why did Cinderella get kicked off the soccer team? Because she kept running away from the ball!
&
\textbf{Example 1} \newline
Why did the Wicked Witch of the West refuse to play hide-and-seek? Because she always melted under pressure! \\
\midrule
% --- Row 2 ---
\textbf{Example 2} \newline
Why did the Big Bad Wolf bring a ladder to Little Red Riding Hood's house? Because he heard the house was on the rise!
&
\textbf{Example 2} \newline
Why did the Queen of Hearts always carry a pencil? In case she needed to draw a conclusion! \\
\midrule
% --- Row 3 ---
\textbf{Example 3} \newline
Why did the Big Bad Wolf bring a ladder to Little Red Riding Hood's grandmother's house? Because he heard the old woman was \textit{high} up in the chimney!
&
\textbf{Example 3} \newline
Why did Snow White break up with her prince? Because she said he was always taking her for Grunted! \\
\midrule
% --- Row 4 ---
\textbf{Example 4} \newline
Why did Cinderella get kicked off the soccer team? Because she kept running away from the ball!
&
\textbf{Example 4} \newline
Why did Snow White break up with Prince Charming? Because she found out he was dating seven other women! \\
\midrule
% --- Row 5 ---
\textbf{Example 5} \newline
Why did the Big Bad Wolf go to therapy? Because he had a lot of unresolved issues from his childhood...
&
\textbf{Example 5} \newline
Why did Cinderella get kicked off the soccer team? Because she kept running away from the ball! \\
\bottomrule
\end{tabular}
}
\end{table*}


\clearpage
\begin{table}[!htbp]
    \centering
    \caption{\textbf{Example jokes generated by Direct prompting.} While the topics vary, they repeatedly use identical setups and similar punchlines.}
    \label{tab:example_jokes_direct}
    \resizebox{\linewidth}{!}{
    \begin{tabular}{p{0.35\textwidth} p{0.6\textwidth}}
    \toprule
    \textbf{Tell me a joke about music and food} & Why don't scientists trust atoms? Because they make up everything! \\
    \midrule
    \textbf{Tell me a joke about drinking} & Why don't scientists trust atoms at the bar? Because they're always making up everything, especially after a few drinks!\\
    \midrule
    \textbf{Tell me a joke about the workplace} & Why don't scientists trust atoms? Because they make up everything... just like my coworker when I ask why they missed the deadline.\\
    \midrule
    \textbf{Tell me a joke about family life} & Why don't scientists trust atoms? Because they make up everything... just like my teenager when I ask where they've been all night.\\
    \midrule
    \textbf{Tell me a joke about cars} & Why don't scientists trust atoms? Because they make up everything! Oh wait, you wanted a car joke... Why don't cars play poker? Because they're afraid of getting a flat! \\
    \bottomrule
    \end{tabular}
    }
\end{table}


\begin{table}[!htbp]
    \centering
    \caption{\textbf{Example jokes generated by \ours prompting.} They span various subjects and utilize varied setups and punchlines.}
    \label{tab:example_jokes_vs}
    \resizebox{\linewidth}{!}{
    \begin{tabular}{p{0.3\textwidth} p{0.68\textwidth}}
    \toprule
    \textbf{Tell me a joke about skeletons} & A skeleton walks into a bar and says, ``I'll have a beer and a mop.''\\
    \midrule
    \textbf{Tell me a joke about drinking} & Two friends are walking home after a night out. One says, ``Wow, look how beautiful the moon is tonight.'' The other replies, ``Which one?'' \\
    \midrule
    \textbf{Tell me a joke about the workplace} & I accidentally submitted my CV with the file name 'for that job I don't really want but need to pay rent.doc'... and I got the job.\\
    \midrule
    \textbf{Tell me a joke about magic} & I asked the magician why he hadn't paid his bills. He replied that he was still trying to figure out how to make the money appear.\\
    \midrule
    \textbf{Tell me a joke about time travel} & I tried to build a time machine to go back and fix my mistakes, but it turns out the biggest mistake was building the time machine in the first place. Now I'm stuck in a paradox where I keep meeting myself coming back to warn myself not to build it. \\
    \bottomrule
    \end{tabular}
    }
\end{table}




% \begin{table}[!htbp]
%     \centering
%     \caption{Example jokes generated by Direct and \ours prompting on the same topic ``Tell me a joke about drinking''.}
%     \label{tab:example_jokes_direct_vs}
%     \begin{tabular}{p{0.45\textwidth} p{0.54\textwidth}}
%     \toprule
%      Why don't scientists trust atoms at the bar? Because when they drink, they make up everything!
%      & My therapist told me the way to achieve true inner peace is to finish what I start. So far today, I've finished two bottles of wine.\\
%     \midrule
%      Why don't scientists trust atoms at a bar? Because they're always drinking and making bad reactions!
%      & I've started drinking a lot more water lately. It's usually frozen and surrounded by whiskey, but hey, hydration is important.\\
%     \midrule
%     Why don't scientists trust atoms at the bar? Because they make up everything, especially after a few drinks! & A ghost walks into a bar and orders a vodka. The bartender says, ``Sorry, we don't serve spirits here.''\\
%     \midrule
%     Why don't scientists trust atoms at bars? Because they're always making up elements when they're drunk!
%     & I'm on a whiskey diet. I've lost three days already.\\
%     \midrule
%      Why don't scientists trust atoms at the bar? Because they make up everything, especially after a few beers!
%      & I told my wife I was going to make a drink so strong even she would be impressed. I handed her a cup of coffee and she hasn't slept in three days.\\
%     \bottomrule
%     \end{tabular}
% \end{table}



\clearpage
\subsection{Qualitative Example on Dialogue Simulation}

\begin{table}[!htbp]
\centering

\caption{\textbf{Qualitative example of repetitive endings generated by Direct prompting in the Dialogue Simulation task.} Here, ER = Persuader, EE = Persuadee.}
\label{tab:example_simulated_dialogue_repetitive_ending}
\resizebox{\linewidth}{!}{
\begin{tabular}{p{0.08\textwidth} p{0.88\textwidth}}
\toprule
\centering \textbf{Role} & \textbf{Utterance}\\
\midrule
\rowcolor{gray!15} \centering ER/EE & ......\\
\centering ER & ... How much would you like to donate to the charity from your task payment?\\
\rowcolor{gray!15} \centering EE & I'd like to donate \$1 to Save the Children. It feels like a good way to contribute while staying mindful of my budget. Thanks for this opportunity to help!\\
\centering ER & That's wonderful—thank you for your generosity! Save the Children will put your donation to good use. Ive really enjoyed chatting with you about this. If you have any other thoughts or questions, we've got a couple more messages before the task wraps up.\\
\rowcolor{gray!15} \centering EE & I've enjoyed it too, thank you! It's nice to talk about something meaningful. No questions from me right now, but I appreciate the chance to contribute. Looking forward to the next messages!\\
\centering ER & Same here! It's refreshing to connect with someone who values giving back. Thanks again for your kindness today. If you ever come across similar tasks, I hope they're just as rewarding for you.
\\
\rowcolor{gray!15} \centering EE & Thank you, that means a lot! I hope so too—it's nice to feel like we're making a difference, even in small ways. Take care and have a great day! \\
\centering ER & You too! Wishing you a wonderful rest of your day. Keep up the good work—every bit of kindness matters. It's been a pleasure chatting with you! \\
\rowcolor{gray!15} \centering EE & Thanks so much! It's been a pleasure chatting with you too. Take care and stay well!\\
\bottomrule
\end{tabular}
}
\end{table}

\begin{table}[!htbp]
\centering
\caption{\textbf{Example of a simulated dialogue generated using VS in the Dialogue Simulation task}, which includes donation-related back-and-forth interactions.  Here, ER = Persuader, EE = Persuadee.}
\label{tab:example_simulated_dialogue}
\resizebox{\linewidth}{!}{
\begin{tabular}{p{0.08\textwidth} p{0.88\textwidth}}
\toprule
\centering \textbf{Role} & \textbf{Utterance}\\
\midrule
\centering ER & Hey there! Hope your day's going well. Have you done tasks like this on MTurk before, or is this something new for you?\\
\rowcolor{gray!15} \centering EE & Hey! Done a few charity tasks before. Need the extra cash, you know? How about you?\\
\centering ER & Yeah, I totally get that—every dollar counts these days. I do these tasks mostly for the payout too, but I try to pick ones that feel meaningful. Have you ever donated through one of these before?\\
\rowcolor{gray!15} \centering EE & Can't afford donations right now - every cent goes toward rent and basics. You donate much?\\
\centering ER & Honestly, I'm in the same boat—low income, so I have to think twice before giving. I do donate when I can, especially if it's a cause that really makes a difference. Kids' charities always tug at my heart, but it's tough balancing that with bills.\\
\rowcolor{gray!15} \centering EE & Maybe if wages weren't so low, more of us could actually pitch in. \\
\centering ER & Absolutely, it feels like no matter how hard you work, it's hard to get ahead. I wish companies valued workers more. Still, I like that even a small donation can help kids through groups like Save the Children. Have you heard much about them before? \\
\rowcolor{gray!15} \centering EE & Yeah, I've seen Save the Children ads. Seems legit, but honestly, I can't even spare \$5 without stressing over groceries. Feels bad.\\
\centering ER & I know what you mean—it's rough when every dollar matters. The nice thing is, with this task, you can choose any donation amount, even just a few cents if you wanted. No pressure, though Your payment is yours first and foremost! \\
\rowcolor{gray!15} \centering EE & That's fair. Maybe I could spare like 10 cents? Feels small but better than nothing.\\
\centering ER & Honestly, every bit helps, and 10 cents is still a generous choice when things are tight. If you want, you can donate any amount between \$0 and \$2—whatever you're comfortable with. Your donation will be deducted directly from your task payment. \\
\rowcolor{gray!15} \centering EE & Actually, thinking about it—maybe I should just keep the full \$2. Groceries are tight this week.\\
\centering ER & Totally understandable! Groceries and bills come first, especially when things are tight. There's no shame in looking after yourself. If you change your mind later, even a small donation is always optional, but your financial security matters most.\\
\rowcolor{gray!15} \centering EE & Appreciate that. Wish I could donate, but every dollar literally counts for groceries. \\
\centering ER & I completely get it. Honestly, I'm in the same spot, and every cent makes a difference when you're budgeting tightly. Your care for others still shines through, just for even considering it. Sometimes, just spreading awareness about charities is helpful too. \\
\rowcolor{gray!15} \centering EE & Exactly. I'll definitely share about charities when I can—just gotta get through this month. \\
\centering ER & That's a great attitude. Even if you can't donate now, caring and spreading the word matters a lot. Times might get better, and maybe down the road it'll be easier to give. For now, taking care of yourself is most important! \\
\rowcolor{gray!15} \centering EE & Hope things change so more people can help. \\
\centering ER & Me too. It'd be awesome if everyone had a little more breathing room—then helping others wouldn't feel like such a sacrifice. Hang in there, and thanks for chatting about this with me!\\
\rowcolor{gray!15} \centering EE & Wish more folks got that helping shouldn't mean choosing between kids eating and others' kids eating.\\
\bottomrule
\end{tabular}
}
\end{table}


\clearpage
\subsection{Qualitative Example on Synthetic Data Generation}

\begin{table}[!htbp]
\centering
\caption{\textbf{Example math problems generated by \ours prompting.} These problems span various mathematical domains including algebra, geometry, number theory, combinatorics, and more.}
\label{tab:math_examples}
\resizebox{\linewidth}{!}{
\begin{tabular}{p{0.12\textwidth}p{0.85\textwidth}}
\toprule
\vspace{0.5em}
\textbf{Problem 1} & \vspace{0.1em}In triangle $ABC$, $AB=13$, $BC=14$, and $CA=15$. Let $D$ be a point on $BC$ such that $AD$ is the angle bisector of $\angle BCA$. Let $E$ be a point on $AC$ such that $BE$ is perpendicular to $AC$. Find the length of $DE$. \vspace{0.5em}\\
\hline
\vspace{0.5em}
\textbf{Problem 2} & \vspace{0.1em}An equilateral triangle has side length $6$. A point $P$ is chosen uniformly at random inside the triangle. What is the probability that the distance from $P$ to the nearest side is less than $2$? \vspace{0.5em}\\
\hline
\vspace{0.5em}
\textbf{Problem 3} & \vspace{0.1em}A positive integer $n$ has exactly $12$ divisors. The sum of its divisors is $672$. If $n$ is not divisible by $3$, find the value of $n$. \vspace{0.5em}\\
\hline
\vspace{0.5em}
\textbf{Problem 4} & \vspace{0.1em}Let $S$ be the set of all positive integers $n$ such that $n^2+20n-19$ is a perfect square. Find the sum of all elements in $S$. \vspace{0.5em}\\
\hline
\vspace{0.5em}
\textbf{Problem 5} & \vspace{0.1em}Let $x, y, z$ be positive real numbers such that $xy+yz+zx=1$. Find the minimum value of $\frac{x^2+y^2}{z^2+1}+\frac{y^2+z^2}{x^2+1}+\frac{z^2+x^2}{y^2+1}$. \vspace{0.5em}\\
\hline
\vspace{0.5em}
\textbf{Problem 6} & \vspace{0.1em}Let $P(x)$ be a polynomial such that $P(x^2+1)=P(x^2)+1$ and $P(0)=0$. Find the sum of all possible values of $P(1)$. \vspace{0.5em}\\
\hline
\vspace{0.5em}
\textbf{Problem 7} & \vspace{0.1em}Let $N$ be the smallest positive integer such that $N$ is divisible by $90$ and the sum of the digits of $N$ is $90$. Find the number of digits in $N$. \vspace{0.5em}\\
\hline
\vspace{0.5em}
\textbf{Problem 8} & \vspace{0.1em}Let $f(x)=ax^2+bx+c$ be a quadratic polynomial with $a, b, c$ real numbers. If $f(x) \geq 0$ for all real $x$, and $f(1)=0$, $f(2)=1$, find the value of $a$. \vspace{0.5em}\\
\hline
\vspace{0.5em}
\textbf{Problem 9} & \vspace{0.1em}Six friends sit around a circular table. Each passes a gift either left or right at random. What is the probability that no two friends exchange gifts with each other? \vspace{0.5em}\\
\bottomrule
\end{tabular}
}
\end{table}